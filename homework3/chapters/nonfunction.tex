\section{系统非功能性需求分析}
\subsection{性能}
\textbf{时延}
直播系统的时延主要体现在主播将视频流推送至媒体服务器以及用户对该直播频道的观看上。用户向直播网站发送主播视频请求以及直播网站响应用户的请求,
这个时间应该在用户可接受的范围之内,一般正常人的反应时间为0.2s,而用户打开一个网页耐心等待时间不超过5s,超过这个时间,
则当前网页极有可能被用户停止访问,因此从主播视频流推送至服务器,到用户访问直播网站 观看直播的时延介于0.2s〜5s之间最为合适。

\textbf{稳定性}
在CDN系统中,服务器集群应该具有很好的稳定性,不因用户的访问量的增多而使得整个服务器集群系统负载量分布不均衡。
标准差能反映一个数据集的离散程度,对于整个服务器集群的负载量分布来说,标准差越小越好。并且直播系统直接面向老师和学生,
一旦有多人同时使用,必将带来高流量、大带宽的访问量,随着访问量的增大,
系统也能有条不紊地继续为用户提供服务,同时还需要容纳一些常规性错误,系统的稳定性是保证系统有条不紊提供服务的前提。

\textbf{同步性}
实时的在线直播系统不管是在音视频传输还是涂鸦操作上,对同步性的要求都会很高,一旦无法满足实时性,在上课效果上会有很大的影响,
系统设计是必须满足很高的同步性需求,来支持更好的上课效果。

\textbf{资源使用率}
\begin{enumerate}
\item CPU占用率 $\leq$50\%。
\item 内存占用率 $\leq$50\%。
\end{enumerate}

\subsection{安全性}
\begin{enumerate}
\item 严格权限访问控制,用户在经过身份认证后,只能访问其权限范围内的数据,只能进行其权限范围内的操作。
\item 不同的用户具有不同的身份和权限,需要在用户身份真实可信的前提下,提供可信的授权管理服务,保护数据不被非法/越权访问和篡改,要确保数据的机密性和完整性。
\item 提供运行日志管理及安全审计功能,可追踪系统的历史使用情况。
\item 能经受来自互联网的一般性恶意攻击。如病毒(包括木马)攻击、口令猜测攻击、黑客入侵等。
\item 至少99\%的攻击需要在10秒内检测到。
\end{enumerate}

\subsection{可靠性}
雨课堂系统要求能在 24 小时内一直稳定运行。当系统处于正常上课时间,会有大流量的客户涌入,对系统的高并发性和可靠性要求较高,如果只有一台应用服务器来接受来自很多客户端的直播请求,一旦资源耗尽,发生宕机的可能性很大,所以系统的可靠性十分重要。

\textbf{易恢复性}
本系统发生故障后,系统应重建其性能水平并恢复直接受影响数据的能力。发布新版本时,应要做好回滚方案,以备异常紧急处理。同时做好备份,系统监控的字段以及历史查询信息误删除时可进行恢复。

\begin{table}[]
\begin{tabular}{lll}
使用时段 &
  数据恢复目标(RPO) &
  恢复时间目标(RTO) \\
\begin{tabular}[c]{@{}l@{}}高峰时段:每天8:00-22:00\\    \\ 运行时段:5:00-次日2:00\\    \\ 系统维护时间:2:00-5:00\end{tabular} &
  回滚至故障发生之前最近的一次日志记录点。数据每隔1min记录一次,发生故障回滚至故障发生1min前。 &
  5min-30min
\end{tabular}
\end{table}

\textbf{容错性}
在系统出错时,不影响用户的行为操作与数据。在设计数据库的时候,应进行冗余设计,采用主从数据库的方式,把读操作和写操作分离,部署在不同的服务器上,从数据库主要用来查询数据用,不进行写入操作,主数据库为写库,用来写入和更新数据,每次当主数据库有写的操作时,数据同步到从数据库去。设计多个从数据库,即拥有了多个容灾副版本,当主数据库服务器宕机的时候,马上切换到其中一台正常运行的从数据库服务器去,以整个系统的容错性。

\textbf{成熟性}
系统故障率需要保持在百分之一以下。系统应成熟度较高,错误防御机制和错误处理机制较为完备成熟。

\subsection{易用性}
易用性是以用户为中心,结合视觉、交互、情感等综合感受,使产品符合用户习惯的能力以及其对使用的期望。 它会对用户使用产品的生产效率、错误率以及用户对新产品的接收程度产生很大的影响。

\textbf{易学习性}
系统学习成本低。雨课堂系统应无需学习即可使用。功能入口清晰,方便操作,操作流程合理。

\textbf{易操作性}
系统建设过程中,系统应涵盖完整的业务需求。各个功能模块:班级管理、直播、课后回放、课堂互动等模块之间实现信息的顺畅流动,系统具有连贯性,交互设置合理,功能明确清晰。

\textbf{用户错误防御机制}
雨课堂系统应具有多种错误防御机制,容错能力强。包括:
\begin{enumerate}
\item 系统遇到错误的输入时会触发检测机制,提示用户输入错误,防止造成系统崩溃影响用户使用体验。
\item 系统应具有成熟的并发控制,在高并发大流量下有一系列的拥塞控制处理机制。
\item 扩展插件系统:扩展多种教学课件插件,支持多种格式的文件上传。
\end{enumerate}


\textbf{用户界面美观}
% Please add the following required packages to your document preamble:
% \usepackage{multirow}
\begin{table}[]
\begin{tabular}{lll}
需求分类                  & 定义要素   & 需求内容                                           \\
\multirow{4}{*}{用户界面} & 界面风格要求 & 简约、风格简明清晰、适合学生                                 \\
                      & 界面导航要求 & 功能划分清晰,导航简明清楚                                  \\
                      & 界面输入要求 & 支持多种格式插件,教学课件类型的扩展丰富灵活                         \\
                      & 界面提示要求 & 系统设计了针对教学的多种功能性提示,包括上课提醒、作业截止日期提醒、教师发布新消息提醒等等。
\end{tabular}
\end{table}

\subsection{可维护性与可扩展性}
在设计与开发本系统时,需达到以下要求:

\begin{enumerate}
\item 程序级别的可扩展性,主要通过参数化配置程序低级别的可扩展性。
\item 高度可配置性,包括各种属性文件和配置文件。
\item 插件系统丰富。插件系统能支持多种教学课件的放映,支持多种格式的文件上传与下载。
\end{enumerate}
	
为使系统具有可维护性与可扩展性,要求系统具有:

\textbf{模块性}
雨课堂业务流程变化较多,此时将系统按功能进行模块化划分,支持灵活配置,有利于减少重复开发量。雨课堂系统应高度内聚各个功能模块,通过抽象类设计、接口设计等方式减少模块的耦合。引用外部方法时,将外部方法作为变量传入本函数作用域,再加以使用。同时,本系统在设计每个功能模块时,如班级管理、课堂直播、课堂互动、课下作业与通知管理等等,尽可能做到了划分明确,降低彼此间的依赖性。

\textbf{可复用性}
课堂互动中的子模块(发送弹幕、投票、小测、点名等)以及模块功能类似,系统开发时应进行统一设计,在需要用到的地方可进行微调然后调用,开发过程中应注重提高模块的可重用性。

\textbf{易分析性}
系统应每隔一段时间生成系统运行日志,可追踪系统的历史使用情况,易诊断缺陷或失败原因。
