\section{系统功能性需求分析}
\subsection{直播模块}

 系统目前不支持网络直播,无法满足当前形势下教学场景的真实诉求。为此,系统应该增加网络直播功能模块,其中包括老师开播,记录学生进入时间及学生在线时间,直播回放和互动面板的功能。

\begin{figure}[!htb]
	\centering\label{fig:fig1}
	\includegraphics[scale=0.6]{image/f1.png}
	\caption{直播模块用例图}
\end{figure}



\begin{longtable}[c]{c|ccc}
	\caption{教师开播用例表}
	\label{tab:tab1}\\
	\shline
	\multicolumn{1}{c|}{\textbf{用例编号}} & \multicolumn{1}{c|}{UC-01} & \multicolumn{1}{c|}{用例名称} &  教师开播\\ \hline
	\endhead
	%
	\multicolumn{1}{c|}{\textbf{活动者}} & \multicolumn{1}{c|}{用户(教师)} & \multicolumn{1}{c|}{优先级} &高  \\ \hline
	\textbf{用例描述} & \multicolumn{3}{p{12cm}}{该用例用来描述老师在预定的课程时间进入班级的直播模块,开启直播。} \\ \hline
	\textbf{前置条件}& \multicolumn{3}{p{12cm}}{教师在预定的课程时间进入班级。} \\ \hline
	\textbf{基本事件流}& \multicolumn{3}{p{12cm}}{教师在预定的课程时间进入班级界面;\newline
	    教师进入班级界面后进入直播界面;\newline
	    点击开播,开始直播。
	} \\ \hline
	\textbf{异常事件流}& \multicolumn{3}{p{12cm}}{1.教师未在预定时间开播;\newline
	2.教师在进入直播模块后未显示开播按钮。
	} \\ \hline
	\textbf{后置条件}& \multicolumn{3}{p{12cm}}{教师开播成功并通知该课程学生及时上课。} \\ \shline
\end{longtable}

\begin{longtable}[c]{c|ccc}
	\caption{直播回放用例表}
	\label{tab:tab2}\\
	\shline
	\multicolumn{1}{c|}{\textbf{用例编号}} & \multicolumn{1}{c|}{UC-02} & \multicolumn{1}{c|}{用例名称} &  直播回放\\ \hline
	\endhead
	%
	\multicolumn{1}{c|}{\textbf{活动者}} & \multicolumn{1}{c|}{用户} & \multicolumn{1}{c|}{优先级} &中  \\ \hline
	\textbf{用例描述} & \multicolumn{3}{p{12cm}}{该用例用来描述直播结束后,用户可进入直播模块寻找指定课程进行回放。} \\ \hline
	\textbf{前置条件}& \multicolumn{3}{p{12cm}}{教师直播结束。} \\ \hline
	\textbf{基本事件流}& \multicolumn{3}{p{12cm}}{教师直播结束。
	} \\ \hline
	\textbf{异常事件流}& \multicolumn{3}{p{12cm}}{教师直播过程中出现问题提前结束直播。
	} \\ \hline
	\textbf{后置条件}& \multicolumn{3}{p{12cm}}{直播结束后,用户可在已有直播记录中找到指定直播回放观看。} \\ \shline
\end{longtable}

\begin{longtable}[c]{c|ccc}
	\caption{互动面板用例表}
	\label{tab:tab3}\\
	\shline
	\multicolumn{1}{c|}{\textbf{用例编号}} & \multicolumn{1}{c|}{UC-03} & \multicolumn{1}{c|}{用例名称} &  互动面板\\ \hline
	\endhead
	%
	\multicolumn{1}{c|}{\textbf{活动者}} & \multicolumn{1}{c|}{用户} & \multicolumn{1}{c|}{优先级} &高  \\ \hline
	\textbf{用例描述} & \multicolumn{3}{p{12cm}}{该用例用来描述直播间中的成员可以通过互动面板对该直播内容进行讨论和互动} \\ \hline
	\textbf{前置条件}& \multicolumn{3}{p{12cm}}{教师正在直播。} \\ \hline
	\textbf{基本事件流}& \multicolumn{3}{p{12cm}}{
		教师开始直播;\newline
		点击互动面板按钮;\newline
		打开面板,进行互动。
		} \\ \hline
	\textbf{异常事件流}& \multicolumn{3}{p{12cm}}{用户输入的互动格式有问题。
	} \\ \hline
	\textbf{后置条件}& \multicolumn{3}{p{12cm}}{面板不用时最小化隐藏,回到主直播界面。} \\ \shline
\end{longtable}


\subsection{文件管理}

在每个课堂下添加新的文件管理模块,用于对该课堂的文件分类及整理。

\begin{figure}[!htb]
	\centering\label{fig:fig2}
	\includegraphics[scale=0.5]{image/f2.png}
	\caption{文件管理用例图}
\end{figure}

\begin{longtable}[c]{c|ccc}
	\caption{文件管理用例表}
	\label{tab:tab5}\\
	\shline
	\multicolumn{1}{c|}{\textbf{用例编号}} & \multicolumn{1}{c|}{UC-05} & \multicolumn{1}{c|}{用例名称} &  文件管理\\ \hline
	\endhead
	%
	\multicolumn{1}{c|}{\textbf{活动者}} & \multicolumn{1}{c|}{用户} & \multicolumn{1}{c|}{优先级} &高  \\ \hline
	\textbf{用例描述} & \multicolumn{3}{p{12cm}}{该用例用来描述用户进入课堂课件界面后,按文件类型分类文件,便于用户查找文件。} \\ \hline
	\textbf{前置条件}& \multicolumn{3}{p{12cm}}{用户进入课程课件界面。} \\ \hline
	\textbf{基本事件流}& \multicolumn{3}{p{12cm}}{用户点击课件模块;\newline
	用户界面对文件类型分类,点击某一类型,展示该课堂该类型文件。
	} \\ \hline
	\textbf{异常事件流}& \multicolumn{3}{p{12cm}}{文件分类出错,导致文件难以查找。
	} \\ \hline
	\textbf{后置条件}& \multicolumn{3}{p{12cm}}{从管理界面退出后返回课件主界面。} \\ \shline
\end{longtable}
