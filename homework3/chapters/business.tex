\section{系统业务需求}
受新冠病毒疫情影响,全国大中小学生都不能够及时返校上课,“线上教学”这一概念应运而生。和往常不同的是,曾经“在线教育”不过是互联网时代下的各种对于教育的创新尝试,而现在成为了学生、家长、学校、老师对于学习的必要手段。与传统教学相比,线上教学具有以下优点:
\subsection{个人用户需求}

\begin{enumerate}
\item	打破了线下教学的场地租金限制和生源流量限制,充分利用了多媒体的优势。

平时的线下教育,有地理限制和师资限制,这使得教育公平难以实现,而在线教育,可以有机会实现相对的教育公平。在现实生活中,老师有很多,而好老师很少,学校有很多,而好学校太少,在线教育,能够使得更多人有机会接触到优质的教育资源,甚至能够根据自己的需求而自由组合想要学习的课程,不需要受到现实生活的约束。

\item	线上教学打破时间和空间的约束。

传统的线下教学,需要有固定的时间和空间。一个班级,如果总是有一两个人临时有事请假不来上课,那么整个班级的教学进度和教学质量都会受到影响,而且,线下课堂也需要实体空间,需要固定的教室或实验室。而线上教学,打破了空间的局限,这能让许多与教学内容相关的活动可以更加自由地进行。

\item	线上教学可以让更多的思维得到碰撞,提高知识的新陈代谢。

线上教学使得线下课堂的物理边界消失,因此,师资得到更多的共享,而学生,也变得越来越多元。在这里,我们可以看到,老师也会存在优胜劣汰,老师会在众多的学生面前展示自己,学生也会提出各种问题,而教学内容,也将加速更新进化。而不会像过去,一份教案可以讲很多年,一本教材,可以很多年不变。所以,这也逼得老师们改进自己的陈旧的过时的教学方法。
\end{enumerate}


面对突然大量增长的用户数量,原本作为智慧教学工具用于全面提升课堂教学体验的雨课堂在功能及并发性等方面遇到了前所未有的挑战。主要出现以下问题:
\begin{enumerate}

\item	当系统中同时开设直播在线课堂,一个课堂的在线学习人数较多时,系统随课堂容量的变化而出现卡顿等系统性能下降。
\item	针对教学这一特定领域直播功能缺陷,如直播过程不能有效互动,无针对直播过程的统计分析等等。
\item	基于PPT插件的方式,限制了教学课件类型的扩展,限制了教学内容呈现的灵活性。
\end{enumerate}

面对变化的用户需求,当前用于直播的系统(例如腾讯会议、钉钉等)都是通用型系统,对课堂教学缺乏针对性的支持。因此,若想抓住机遇,有必要及时开发一款针对网络直播课堂的应用,围绕大中小学的教学需求提供针对性网络直播课堂服务,用以满足大容量直播课堂需求。
